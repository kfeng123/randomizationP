\documentclass[review]{elsarticle}

\usepackage{lineno,hyperref}

\modulolinenumbers[5]

\journal{Journal of \LaTeX\ Templates}

%%%%%%%%%%%%%%%%%%%%%%%
%% Elsevier bibliography styles
%%%%%%%%%%%%%%%%%%%%%%%
%% To change the style, put a % in front of the second line of the current style and
%% remove the % from the second line of the style you would like to use.
%%%%%%%%%%%%%%%%%%%%%%%

%% Numbered
%\bibliographystyle{model1-num-names}

%% Numbered without titles
%\bibliographystyle{model1a-num-names}

%% Harvard
%\bibliographystyle{model2-names.bst}\biboptions{authoryear}

%% Vancouver numbered
%\usepackage{numcompress}\bibliographystyle{model3-num-names}

%% Vancouver name/year
%\usepackage{numcompress}\bibliographystyle{model4-names}\biboptions{authoryear}

%% APA style
%\bibliographystyle{model5-names}\biboptions{authoryear}

%% AMA style
%\usepackage{numcompress}\bibliographystyle{model6-num-names}

%% `Elsevier LaTeX' style
%\bibliographystyle{elsarticle-num}
%%%%%%%%%%%%%%%%%%%%%%%
\usepackage{xeCJK}
\usepackage{bm}
\usepackage{amsmath}
\usepackage{amssymb}
\usepackage{amsthm}
\usepackage{graphicx}
\usepackage{color}
\usepackage{booktabs}
\usepackage{algorithm}
\usepackage{algorithmic}




\theoremstyle{plain}
\newtheorem{theorem}{\quad\quad Theorem}
\newtheorem{proposition}{\quad\quad Proposition}
\newtheorem{corollary}[theorem]{Corollary}
\newtheorem{lemma}{Lemma}
\newtheorem{example}{Example}
\newtheorem{assumption}{\quad\quad Assumption}
\newtheorem{condition}{Condition}

\theoremstyle{definition}
\newtheorem{remark}{\quad\quad Remark}
\theoremstyle{remark}




\begin{document}

\begin{frontmatter}

\title{Elsevier \LaTeX\ template\tnoteref{mytitlenote}}
\tnotetext[mytitlenote]{Fully documented templates are available in the elsarticle package on \href{http://www.ctan.org/tex-archive/macros/latex/contrib/elsarticle}{CTAN}.}

%% Group authors per affiliation:
\author{Elsevier\fnref{myfootnote}}
\address{Radarweg 29, Amsterdam}
\fntext[myfootnote]{Since 1880.}

%% or include affiliations in footnotes:
\author[mymainaddress,mysecondaryaddress]{Elsevier Inc}
\ead[url]{www.elsevier.com}

\author[mysecondaryaddress]{Global Customer Service\corref{mycorrespondingauthor}}
\cortext[mycorrespondingauthor]{Corresponding author}
\ead{support@elsevier.com}

\address[mymainaddress]{1600 John F Kennedy Boulevard, Philadelphia}
\address[mysecondaryaddress]{360 Park Avenue South, New York}

\begin{abstract}
This template helps you to create a properly formatted \LaTeX\ manuscript.
\end{abstract}

\begin{keyword}
\texttt{elsarticle.cls}\sep \LaTeX\sep Elsevier \sep template
\MSC[2010] 00-01\sep  99-00
\end{keyword}

\end{frontmatter}

\linenumbers


\section{Introduction}

Consider i.i.d.\ random sample $X_{1},\ldots,X_{n}\in \mathbb{R}^p$ which has means $\mu={(\mu_1,\ldots,\mu_p)}^T$ and covariance matrix $\Sigma$. We consider testing the following high-dimensional hypothesis:
\begin{equation}
    H_0:\mu=0_p\quad \textrm{versus} \quad H_1:\mu\neq 0_p.
\end{equation}

We assume, like chen qin and bai, the following multivariate model:
\begin{equation}
    X_i=\mu+\Gamma Z_i\quad \textrm{for}\, i=1,\ldots,n,
\end{equation}
where $\Gamma$ is a $p\times m$ matrix for some $m\geq p$ such that $\Gamma_i\Gamma_i^T=\Sigma_i$ and ${\{Z_{i}\}}_{i=1}^n$ are $m$-variate i.i.d.\ random vectors satisfying $\mathrm{E}(Z_i)=0$ and $\mathrm{Var}(Z_i)=I_m$, the $m\times m$ identity matrix. Write $Z_i={(z_{i1},\ldots,z_{im})}^T$, we assume $\mathrm{E}(z_{ij}^4)=3+\Delta<\infty$ and
\begin{equation}
    \mathrm{E}(z_{il_1}^{\alpha_1}z_{il_2}^{\alpha_2}\cdots z_{il_q}^{\alpha_q})=\mathrm{E}(z_{il_1}^{\alpha_1})\mathrm{E}(z_{il_2}^{\alpha_2})\cdots \mathrm{E}(z_{il_q}^{\alpha_q})
\end{equation}
for a positive integer $q$ such that $\sum_{l=1}^q \alpha_l\leq 8$ and $l_1\neq l_2\neq \cdots \neq l_q$.

Consider the test statistic 
\begin{equation}
    T(X_1,\ldots,X_n)=\sum_{j<i}X_i^TX_j.
\end{equation}

Statistics like this are studied in some high dimensional mean test literature. Most of existing papers determined the critical value by asymptotic distribution, which is not exact.

\paragraph{Randomization}
Suppose $\epsilon_1,\ldots,\epsilon_n$ are i.i.d. Rademacher variables ($\Pr(\epsilon_i=1)=\Pr(\epsilon_i=-1)=1/2$) which are independent of the data.
Suppose under the null hypothesis is symmetric about $0$, then the conditional distribution
\begin{equation*}
    \mathcal{L}(T(\epsilon_1 X_1,\ldots,\epsilon_i X_i,\ldots,\epsilon_n X_n)|X_1,\ldots,X_n)
\end{equation*}
is the uniform distribution on $2^n$ values.
The critical value of the randomization test is defined as the $1-\alpha$ quantile of the above distribution.
 More specifically, the test function equals to $1$ or $0$ if $T(X_1,\ldots, X_n)$ is greater or not greater than the critical value. The resulting test is a level $\alpha$ test.
If the test function equals to a certain value in $(0,1)$ when $T(X_1,\ldots,X_n)$ is exactly equal to the critical value,  the test is exact.
Since such extreme case occurs with little probability, in practice the former procedure is often adopted, which only drop minor power.



It's easy to see that the randomization test based on $\|\bar{X}\|^2$ , 
$\|\bar{X}\|^2-\frac{1}{n}S$ ($S$ is the sample covariance matrix) and $\sum_{j<i}X_i^T X_j$ are equivalent. Hence the randomization version of Bai's test and Chen's test are equivalent. There's no need to estimate the variance of the statistic.


Randomization method has the advantages that the test is exact when the null distribution is symmetric. A question is that if the ranodomization test can still work if the symmetry condition breaks down.
As far as we know, there's no theoretical work concerning the asymptotic property of randomization test in high dimension setting.
Our asymptotic results show that even the null distribution is not symmetric, the randomization test is still asymptotically exact. 
The local asymptotic power is also given.


In low dimensional setting, it's well known that randomization test is very slow compared to usual method. 
Surprisingly, in high dimension setting the randomization test can be implemented as efficiently as asymptotic method.

Our proof method and computation methods can be used to many other quadratic form based test statistics.

\section{CLT}
\paragraph{CLT for quadratic form of Rademacher varaibles}
We will study the quadratic form of Rademacher variables.
 Let $\epsilon_1,\ldots,\epsilon_n$ be indepent Rademacher  variables. 
 Consider quadratic form $W_n=\sum_{1\leq j<i\leq n} a_{ij}\epsilon_i \epsilon_j$, where $\{a_{ij}\}$ are nonrandom numbers. Here $\{\epsilon_i\}$ and $\{a_{ij}\}$ may depend on $n$, a parameter we suppress.
 Obviously, $\mathrm{E}(W_n)=0$ and $\mathrm{Var}(W_n)=\sum_{1\leq j<i\leq n} a_{ij}^2$.

 \begin{proposition}\label{CLTprop}
     A sufficient condition for
     \begin{equation}
         \frac{W_n}{\sqrt{\sum_{1\leq j<i\leq n} a_{ij}^2}}\xrightarrow{\mathcal{L}} N(0,1)
     \end{equation}
     is that
     \begin{equation}
         \sum_{j<k}{(\sum_{i:i>k}a_{ij}a_{ik})}^2+
         \sum_{j<i}a_{ij}^4+
         \sum_{j<k<i}a_{ij}^2 a_{ik}^2
         =o\big({(\sum_{j<i} a_{ij}^2)}^2\big).
     \end{equation}
 \end{proposition}

 \begin{proof}
     Define $U_{in} =\epsilon_i \sum_{j=1}^{i-1} a_{ij}\epsilon_j$, $i=2,\ldots,n$, and $\mathcal{F}_{in}=\sigma\{\epsilon_1,\ldots,\epsilon_i\}$, $i=1,\ldots, n$.
     Now $W_n=\sum_{i=2}^n U_{in}$ and $\{U_{in}\}$ is a martingale difference array with respect to $\{\mathcal{F}_{in}\}$. 
     To prove the proposition, we shall verify two conditions (See David Pollard's book):
     \begin{equation}\label{MCLTcondition1}
         \frac{\sum_{i=2}^n \mathrm{E}(U_{in}^2 |\mathcal{F}_{i-1,n})}{\sum_{1\leq j<i\leq n} a_{ij}^2}\xrightarrow{P} 1,
     \end{equation}
     and
     \begin{equation}\label{MCLTcondition2}
         \frac{\sum_{i=2}^n \mathrm{E}\big(U_{in}^2\big\{U_{in}^2>\epsilon \sum_{1\leq j<i\leq n} a_{ij}^2\big\}\big|\mathcal{F}_{i-1,n}\big)}{\sum_{1\leq j<i\leq n} a_{ij}^2}\xrightarrow{P} 0,
     \end{equation}
     for every $\epsilon>0$.

     \paragraph{Proof of~\eqref{MCLTcondition1}}
     Since $\mathrm{E}(U_{in}^2 |\mathcal{F}_{i-1,n})={(\sum_{j=1}^{i-1}a_{ij}\epsilon_j)}^2$, we have
     \begin{equation*}
         \begin{aligned}
\sum_{i=2}^n \mathrm{E}(U_{in}^2 |\mathcal{F}_{i-1,n})
             &=\sum_{i=2}^n \big(\sum_{j=1}^{i-1}a_{ij}\epsilon_j \big)^2\\
             &=\sum_{i=2}^n \big( \sum_{j=1}^{i-1} a_{ij}^2 +2\sum_{j,k:j<k<i} a_{ij}a_{ik}\epsilon_j \epsilon_k \big)\\
             &=\sum_{i=2}^n  \sum_{j=1}^{i-1} a_{ij}^2 +2\sum_{j<k<i} a_{ij}a_{ik}\epsilon_j \epsilon_k.
         \end{aligned}
     \end{equation*}

     But
     \begin{equation*}
         \begin{aligned}
         \mathrm{E}{(\sum_{j<k<i} a_{ij}a_{ik}\epsilon_j \epsilon_k)}^2
             &=
             \mathrm{E}{\big(\sum_{j<k} (\sum_{i:i>k}a_{ij}a_{ik})\epsilon_j \epsilon_k \big)}^2\\
             &=
             \sum_{j<k} (\sum_{i:i>k}a_{ij}a_{ik})^2\\
             &=
             o\big({(\sum_{j<i} a_{ij}^2)}^2\big),
         \end{aligned}
     \end{equation*}
     where the last equality holds by assumption. Hence~\eqref{MCLTcondition1} holds.
     \paragraph{Proof of~\eqref{MCLTcondition2}}
     By Markov's inequality, we only need to prove
     \begin{equation}\label{temp1}
         \frac{\sum_{i=2}^n \mathrm{E}\big(U_{in}^4\big|\mathcal{F}_{i-1,n}\big)}{{\big(\sum_{1\leq j<i\leq n} a_{ij}^2\big)}^2}\xrightarrow{P} 0.
     \end{equation}
     Since the relavant random variables are all positive, we only need to prove~\eqref{temp1} converges to $0$ in mean. But
     \begin{equation*}
         \begin{aligned}
         \sum_{i=2}^n \mathrm{E} U_{in}^4
             &=
             \sum_{i=2}^n \mathrm{E} {(\sum_{j:j<i}a_{ij}\epsilon_j)}^4\\
             &=
             \sum_{i=2}^n \mathrm{E} {(\sum_{j:j<i}a_{ij}^2+2\sum_{j,k:j<k<i}a_{ij}a_{ik}\epsilon_j \epsilon_k)}^2\\
             &=
             \sum_{i=2}^n  \big({(\sum_{j:j<i}a_{ij}^2)}^2+4\mathrm{E}{(\sum_{j,k:j<k<i}a_{ij}a_{ik}\epsilon_j \epsilon_k)}^2 \big)\\
             &=
             \sum_{i=2}^n  (\sum_{j:j<i}a_{ij}^4+6\sum_{j,k:j<k<i}a_{ij}^2 a_{ik}^2)\\
             &=
             \sum_{j<i}a_{ij}^4+6\sum_{j<k<i}a_{ij}^2 a_{ik}^2\\
             &=
             o\big({(\sum_{j<i} a_{ij}^2)}^2\big),
         \end{aligned}
     \end{equation*}
     where the last equality holds by assumption. Hence~\eqref{MCLTcondition2} holds.
 \end{proof}

\section{Asymptotic normality}

\begin{lemma}\label{lemmaQ}
    Suppose $A=(a_{ij})$ is an $m\times m$ positive semi-definite matrix, then
    \begin{equation}
        \mathrm{E} {(Z_i^T A Z_i)}^2\asymp (\mathrm{tr}(A))^2+\mathrm{tr}A^2
    \end{equation}
\begin{proof}
Notice that
\begin{equation}
    \begin{aligned}
        {(Z_i^T A Z_i)}^2
        =&
        (\sum_{j=1}^m a_{jj}z_{ij}^2+2\sum_{k<j}a_{jk}z_{ij}z_{ik})^2\\
        =&
        (\sum_{j=1}^m a_{jj}z_{ij}^2)^2+
        4(\sum_{j=1}^m a_{jj}z_{ij}^2)(\sum_{k<j}a_{jk}z_{ij}z_{ik})+
        4(\sum_{k<j}a_{jk}z_{ij}z_{ik})^2\\
        =&
        \sum_{j=1}^m a_{jj}^2z_{ij}^4+2\sum_{k<j}a_{jj}a_{kk}z_{ij}^2 z_{ik}^2+
        4(\sum_{j=1}^m a_{jj}z_{ij}^2)(\sum_{k<j}a_{jk}z_{ij}z_{ik})\\
        &+
        4(\sum_{k<j}a_{jk}^2z_{ij}^2z_{ik}^2+\sum_{k<j,l<\alpha:\mathrm{card}(\{i,j\}\cap\{l,\alpha\})<2} a_{jk}a_{\alpha l}z_{ij}z_{ik}z_{i\alpha}z_{il})\\
    \end{aligned}
\end{equation}
Hence
\begin{equation}
    \begin{aligned}
        \mathrm{E}{(Z_i^T A Z_i)}^2
        =&
        \sum_{j=1}^n a_{jj}^2 \mathrm{E}z_{ij}^4+2\sum_{k<j}a_{jj}a_{kk}\mathrm{E}(z_{ij}^2 z_{ik}^2)+
        4\sum_{k<j}a_{jk}^2 \mathrm{E}(z_{ij}^2z_{ik}^2)\\
        \asymp &
        \sum_{j=1}^n\sum_{k=1}^n a_{jj}a_{kk}+
        \sum_{j=1}^n\sum_{k=1}^n a_{jk}^2
        ={(\mathrm{tr}(A))}^2+\mathrm{tr}A^2
    \end{aligned}
\end{equation}

\end{proof}
\end{lemma}

\begin{lemma}\label{ratioLemma}
    Suppose condition~\eqref{chenCondition} holds, then
    \begin{equation}
        \frac{\sum_{j<i}{(X_i^T X_j)}^2}{\frac{n(n-1)}{2}\mathrm{tr} (\Sigma+\mu\mu^T)^2}
        \xrightarrow{P}1.
    \end{equation}
\end{lemma}
\begin{proof}
    \begin{equation}
        \begin{aligned}
            \mathrm{E}{(X_i^T X_j)}^2=&
            \mathrm{E}(X_i^T X_j X_j^T X_i)=
            \mathrm{E}(X_i^T (\Sigma+\mu \mu^T) X_i)\\
            =&
            \mathrm{E}\mathrm{tr}((\Sigma+\mu \mu^T) X_i X_i^T)=\mathrm{tr}{(\Sigma+\mu \mu^T)}^2
        \end{aligned}
    \end{equation}
    Hence
    \begin{equation}
        \mathrm{E}\sum_{j<i}{(X_i^T X_j)}^2=\frac{n(n-1)}{2}\mathrm{tr}{(\Sigma+\mu\mu^T)}^2
    \end{equation}
    So we only need to consider the variance. According to $\mathrm{card}(\{i,j\}\cap\{k,l\})=0,1,2$, we have
    \begin{equation}\label{eq:1}
    \begin{aligned}
        &{\big(\sum_{j<i}{(X_i^T X_j)}^2\big)}^2
        =
        \sum_{j<i}{(X_i^T X_j)}^4+
        \sum_{j<i,k<l:\{i,j\}\cap \{k,l\}=\phi}{(X_i^T X_j)}^2{(X_k^T X_l)}^2\\
        &+2\sum_{j<i<k}\big(
        {(X_i^T X_j)}^2{(X_k^T X_i)}^2+
{(X_i^T X_j)}^2{(X_k^T X_j)}^2+
{(X_k^T X_j)}^2{(X_k^T X_i)}^2
        \big)
    \end{aligned}
    \end{equation}


    \begin{equation}
        \begin{aligned}
            {(X_i^T X_j)}^4=&
            {(Z_i^T \Gamma^T \Gamma Z_j+\mu^T \Gamma Z_i+\mu^T \Gamma Z_j+\mu^T \mu)}^4\\
            \leq &
            64\big((Z_i^T \Gamma^T \Gamma Z_j)^4+(\mu^T \Gamma Z_i)^4+(\mu^T \Gamma Z_j)^4+(\mu^T \mu)^4\big)\\
        \end{aligned}
    \end{equation}
    Note that by lemma~\ref{lemmaQ}, we have
    \begin{equation}
        \mathrm{E}(\mu^T \Gamma Z_i)^4=
        \mathrm{E}(Z_i^T \Gamma^T \mu\mu^T \Gamma Z_i)^2
        \asymp  {(\mu^T \Sigma \mu)}^2\leq \lambda_{\max}^2(\Sigma){(\mu^T\mu)}^2.
    \end{equation}
    and apply lemma~\ref{lemmaQ}, we get
    \begin{equation}
        \begin{aligned}
            \mathrm{E}(Z_i^T \Gamma^T \Gamma Z_j)^4=&
        \mathrm{E}(Z_i^T \Gamma^T \Gamma Z_j Z_j^T \Gamma^T \Gamma Z_i)^2\\
            =&
            \mathrm{E}\mathrm{E}\big((Z_i^T \Gamma^T \Gamma Z_j Z_j^T \Gamma^T \Gamma Z_i)^2 | Z_j\big)\\
            \asymp &  \mathrm{E}{(Z_j^T \Gamma^T \Sigma \Gamma Z_j)}^2\\
            \asymp &  (\mathrm{tr}\Sigma^2)^2+\mathrm{tr}\Sigma^4\\
            \leq &  (\mathrm{tr}\Sigma^2)^2+\lambda_{\max}^2(\Sigma)\mathrm{tr}\Sigma^2.
        \end{aligned}
    \end{equation}
    By condition~\eqref{chenCondition}, we have 
    \begin{equation}\label{eq:20170220}
        \mathrm{E}{(X_i^T X_j)}^4=O\Big({\big(\mathrm{tr}(\Sigma+\mu\mu^T)^2\big)}^2\Big)
    \end{equation}
Similarly, we have
    \begin{equation}\label{eq:2}
        \begin{aligned}
            &\mathrm{E}{(X_i^T X_j)}^2{(X_k^T X_i)}^2\\
            = &
            \mathrm{E}\mathrm{E}\big({(X_i^T X_j)}^2{(X_k^T X_i)}^2| X_i\big)\\
            =&
            \mathrm{E}{(X_i^T (\Sigma+\mu\mu^T) X_i )}^2\\
            =&
            \mathrm{E}{(Z_i^T \Gamma^T (\Sigma+\mu\mu^T) \Gamma Z_i+ 2\mu^T (\Sigma+\mu\mu^T)\Gamma Z_i +\mu \Sigma \mu +(\mu^T\mu)^2 )}^2\\
            \leq&
            4\mathrm{E}(Z_i^T \Gamma^T (\Sigma+\mu\mu^T) \Gamma Z_i)^2+ 16\mathrm{E}(\mu^T (\Sigma+\mu\mu^T)\Gamma Z_i)^2 +4(\mu \Sigma \mu)^2 +4(\mu^T\mu)^4 \\
            \asymp&
            (\mathrm{tr}(\Gamma^T (\Sigma+\mu\mu^T)\Gamma))^2+\mathrm{tr}(\Gamma^T (\Sigma+\mu\mu^T)\Sigma(\Sigma+\mu\mu^T)\Gamma)\\
            &+ \mu^T (\Sigma+\mu\mu^T)\Sigma (\Sigma+\mu\mu^T)\mu+(\mu \Sigma \mu)^2 +(\mu^T\mu)^4 \\
            =&
            \mathrm{tr}\Sigma^4+
            {(\mu^T \mu)}^4+
            {(\mathrm{tr}\Sigma^2)}^2+
            2\mathrm{tr}\Sigma^2 (\mu^T \Sigma \mu)+
            3 {(\mu^T \Sigma \mu)}^2\\
            &+
            3\mu^T\Sigma^3 \mu+
            2(\mu^T\Sigma^2 \mu)(\mu^T\mu)+
            (\mu^T\Sigma \mu){(\mu^T\mu)}^2\\
            =&
            O(1)({(\mathrm{tr}\Sigma^2)}^2+(\mu^T\mu)^4)\\
            =&O\Big({\big(\mathrm{tr}(\Sigma+\mu\mu^T)^2\big)}^2\Big)
        \end{aligned}
    \end{equation}
    The last $2$ equality follows by the fact $\mu^T \Sigma^i \mu\leq \lambda_{\max}^i(\Sigma)\mu^T\mu$ ($i=1,2,\ldots$), the condition~\eqref{chenCondition} and the inequality $ab\leq a^p/p+b^q/p$ for $1/p+1/q=1$ and $p,q>0$.
   
   In~\eqref{eq:1}, there are $n(n-1)/2$, $n(n-1)(n-2)(n-3)/4$ and $n(n-1)(n-2)/6$ terms in each summation. Hence
    \begin{equation}
    \begin{aligned}
        \mathrm{E}{\big(\sum_{j<i}{(X_i^T X_j)}^2\big)}^2
            =&\frac{n(n-1)(n-2)(n-3)}{4}{\big(\mathrm{tr}(\Sigma+\mu\mu^T)^2\big)}^2\\
            &+O(1)(\frac{n(n-1)}{2}+n(n-1)(n-2)){\big(\mathrm{tr}(\Sigma+\mu\mu^T)^2\big)}^2
    \end{aligned}
    \end{equation}
Hence $\mathrm{Var}(\sum_{j<i}{(X_i^T X_j)}^2)=o\Big(\big(\mathrm{E}(\sum_{j<i}{(X_i^T X_j)}^2)\big)^2\Big)$, which prove the lemma.
\end{proof}

\begin{lemma}
    Suppose~\eqref{chenCondition} holds. Assume
    \begin{equation} 
        \mu^T \mu=o(\sqrt{\mathrm{tr}\Sigma^2}),
    \end{equation}
then
    \begin{equation}\label{lemma2R1}
        \sum_{j<k}{(\sum_{i:i>k}X_i^T X_j X_i^T X_k)}^2
        =o_P\Big(\big(\frac{n(n-1)}{2}\mathrm{tr}(\Sigma+\mu\mu^T)^2\big)^2\Big)
    \end{equation}
    \begin{equation}\label{lemma2R2}
        \sum_{j<k}{(X_i^T X_j)}^4=o_P\Big(\big(\frac{n(n-1)}{2}\mathrm{tr}(\Sigma+\mu\mu^T)^2\big)^2\Big)
    \end{equation}
    \begin{equation}\label{lemma2R3}
        \sum_{j<k<i}{(X_i^T X_j)}^2{(X_i^T X_k)}^2 =o_P\Big(\big(\frac{n(n-1)}{2}\mathrm{tr}(\Sigma+\mu\mu^T)^2\big)^2\Big)
    \end{equation}
\end{lemma}
\begin{proof}
    \begin{equation}
    \begin{aligned}
        &\mathrm{E}\sum_{j<k}{(\sum_{i:i>k}X_i^T X_j X_i^T X_k)}^2\\
        =&
        \mathrm{E}\sum_{j<k}\Big(\sum_{i:i>k}(X_i^T X_j)^2 (X_i^T X_k)^2+2\sum_{i_1,i_2:i_1>i_2>k}X_{i_1}^T X_j X_{i_1}^T X_k X_{i_2}^T X_j X_{i_2}^T X_k\Big).\\
    \end{aligned}
    \end{equation}

    By~\eqref{eq:2}, we have
    \begin{equation}
    \begin{aligned}
        \mathrm{E}\sum_{j<k<i}(X_i^T X_j)^2 (X_i^T X_k)^2    =O(n^3)\big(\mathrm{tr}(\Sigma+\mu\mu^T)^2\big)^2
    \end{aligned}
    \end{equation}
    But
    \begin{equation}
    \begin{aligned}
        &\mathrm{E}\sum_{j<k<i_2<i_1}X_{i_1}^T X_j X_{i_1}^T X_k X_{i_2}^T X_j X_{i_2}^T X_k\\
        =&\frac{n(n-1)(n-2)(n-3)}{6}\mathrm{tr}{(\Sigma+\mu\mu^T)}^4\\
        \leq& \frac{n(n-1)(n-2)(n-3)}{6}8(\mathrm{tr}{(\Sigma)}^4+(\mu^T \mu)^4)\\
        \leq& O(n^4)(\lambda_{\max}^2(\Sigma)\mathrm{tr}{(\Sigma)}^2+(\mu^T \mu)^4)\\
        =&
        o\Big(n^4{\big(\mathrm{tr}\Sigma^2\big)}^2\Big).
    \end{aligned}
    \end{equation}
    It follows that~\eqref{lemma2R1} holds.~\eqref{lemma2R2} and~\eqref{lemma2R3} can be proved similarly by~\eqref{eq:2} and~\eqref{eq:20170220}.

\end{proof}


As Janssen's paper, let $d$ denote any metric on the set of probability measures $\mathcal{M}_1(\mathbb{R})$ on $\mathbb{R}$ such that convergence in $(\mathcal{M}_1(\mathbb{R}),d)$ is equivalent to weak convergence.
\begin{theorem}
     Suppose
    \begin{equation}\label{chenCondition}
        \frac{\lambda_{\max}^2(\Sigma)}{\mathrm{tr}\Sigma^2}\to 0,
    \end{equation}
    and
    \begin{equation}
        \mu^T\mu=o(\sqrt{\mathrm{tr}\Sigma^2}),
    \end{equation}
    then
    \begin{equation}
        d\Big(\mathcal{L}\Big(\frac{T_2(\epsilon_1 X_1,\ldots, \epsilon_i X_i,\ldots,\epsilon_n X_n)}{\sqrt{\sum_{1\leq j<i\leq n}{(X_i^T X_j)}^2}}\Big|X_1,\ldots,X_n\Big),
        N(0,1)\Big)
        \xrightarrow{P}0.
    \end{equation}
\end{theorem}
\begin{proof}
    Similar to the proof of Chen's theorem and by Lemma~\ref{ratioLemma}, we have
    \begin{equation}
        \frac{T_2(X_1,\ldots,X_n)-\frac{n(n-1)}{2}\mu^T\mu}{\sqrt{\sum_{1\leq j< i\leq n}(X_i^T X_j)^2}}\xrightarrow{\mathcal{L}}N(0,1).
    \end{equation}
    For every subsequence of $\{n\}$, there is a further subsequence $\{n(k)\}$ along which~\eqref{lemma2R1},~\eqref{lemma2R2} and~\eqref{lemma2R3} holds almost surely.
    By Proposition~\ref{CLTprop}, we have
    \begin{equation}
        \mathcal{L}\Big(\frac{T_2(\epsilon_1 X_1,\ldots, \epsilon_i X_i,\ldots,\epsilon_n X_n)}{\sqrt{\sum_{1\leq j<i\leq n}{(X_i^T X_j)}^2}}\Big|X_1,\ldots,X_n\Big)\xrightarrow{\mathcal{L}}N(0,1)
    \end{equation}
    almost surely.
    Denote by $\xi_{\alpha}^*$ the $1-\alpha$ quantile of the above conditional distribution, then along $\{n(m)\}$ we have $\xi_{\alpha}^*\to \Phi(1-\alpha)$ almost surely, where $\Phi(\cdot)$ is the CDF of standard normal distribution. Then $\xi^*_{\alpha}\xrightarrow{P}\Phi(1-\alpha)$.
    Now the asymptotic power function can be derived by Slutsky's theorem
    \begin{equation}
        \begin{aligned}
            &\Pr\Big(\frac{T_2( X_1,\ldots, X_n)}{\sqrt{\sum_{1\leq j<i\leq n}{(X_i^T X_j)}^2}}>\xi_{\alpha}^* \Big)\\
            =&
            \Pr\Big(\frac{T_2( X_1,\ldots, X_n)-\frac{n(n-1)}{2}\mu^T\mu}{\sqrt{\sum_{1\leq j<i\leq n}{(X_i^T X_j)}^2}}>\xi_{\alpha}^*-\frac{\frac{n(n-1)}{2}\mu^T\mu}{\sqrt{\sum_{1\leq j<i\leq n}{(X_i^T X_j)}^2}} \Big)\\
            =&
            \Pr\Big(\frac{T_2( X_1,\ldots, X_n)-\frac{n(n-1)}{2}\mu^T\mu}{\sqrt{\frac{n(n-1)}{2}\mathrm{tr}\Sigma^2}}-
            \frac{\sqrt{\sum_{1\leq j<i\leq n}{(X_i^T X_j)}^2}}{\sqrt{\frac{n(n-1)}{2}\mathrm{tr}\Sigma^2}}\xi_{\alpha}^*>
            -\frac{\sqrt{n(n-1)}\mu^T\mu}{\sqrt{2\mathrm{tr}\Sigma^2}} \Big)\\
            =&
            \Pr\Big(N(0,1)-\Phi(1-\alpha)>-\frac{\sqrt{n(n-1)}\mu^T\mu}{\sqrt{2\mathrm{tr}\Sigma^2}}\Big)+o(1)\\
            =&
            \Phi(-\Phi(1-\alpha)+\frac{\sqrt{n(n-1)}\mu^T\mu}{\sqrt{2\mathrm{tr}\Sigma^2}})+o(1),
        \end{aligned}
    \end{equation}
    where the last two equality holds because an exersize of durrett.



\end{proof}


\section{Algorithm}

The quantile might not be computationally feasible, since the randomized test statistic is (conditionally) uniformly distributed on $2^n$ different values.
In practice, randomization test is often realized through an approximation of $p$-value.
More specifically, first choose a large integer $M$.
We can sample from conditional distribution of randomized statistic by generate $\epsilon_1,\ldots,\epsilon_n$ and compute $T(\epsilon_1 X_1,\ldots,\epsilon_n X_n)$.
Repeat $M$ times and we obtain $T_i^*$, $i=1,\ldots,M$.
Denote $\xi_i=\mathbf{1}_{\{T_i^*\geq T_0\}}$.
Then  $\sum_{i=1}^M \xi_i/M$ is an approximation of the $p$-value 
$$p(X_1,\ldots,X_n)=\Pr(T_i^*\geq T_0|X_1,\ldots,X_n).$$
Hence we reject the null if $\sum_{i=1}^M \xi_i/M\leq \alpha$.

Sometimes we can stop and draw a conclusion before sampling $M$ times.
In fact, we can accept the null once the sum of $\xi_i$ exceeds $M\alpha$ and reject the null once the sum of $1-\xi_i$ reaches $M(1-\alpha)$.

Note that once we have obtained $X_i^T X_j$, the computation of $T_i^*$ has only complexity $O(n^2)$.
The total complexity is $O(n^2 (p+M))$, which is fast.
This is different from low dimensional setting where randomization is a lot slower than asymptotic method.
For example, when $M=p$, time spent by randomization is at most twice of that of asymptotic method.



%We shall choose $M$ large enough such that 
%$$\Pr\Big(\Big|\frac{1}{M}\sum_{i=1}^M \xi-p(X_1,\ldots,X_n)\Big|>t|X_1,\ldots,X_n\Big)$$
%is smaller than $\epsilon$, where $t$ and $\epsilon$ are specified. By Hoeffding's inequality,
%\begin{equation*}
    %\Pr\Big(\Big|\frac{1}{M}\sum_{i=1}^M \xi-p(X_1,\ldots,X_n)\Big|>t|X_1,\ldots,X_n\Big)\leq 2e^{-2Mt^2}.
%\end{equation*}
%Hence we choose $M=\big[\frac{1}{2t^2}\log(\frac{2}{\epsilon})\big]+1$.






\begin{algorithm}
    \caption{Randomization Algorithm}
\label{theAlgorithm}
    \begin{algorithmic}
        \REQUIRE  $\alpha$, $M$
        \STATE Set $A\gets 0$, $T_0\gets T(X_1,\ldots,X_n)$.
        \STATE Compute $X_i^T X_j$ for $1\leq j< i\leq n$
        \FOR{$i=1$ to $M$}
            \STATE Generate $\epsilon_1,\ldots,\epsilon_n$ and compute $T(\epsilon_1 X_1,\ldots,\epsilon_n X_n)=\sum_{j<i} X_i^T X_j\epsilon_i\epsilon_j$.
            \IF{$T(\epsilon_1 X_1,\ldots,\epsilon_n X_n)>T_0$}
            \STATE $A\gets A+1$
            \ENDIF
            \IF{$A> M\alpha$}
            \RETURN{Accept}
            \ENDIF
            \IF{$i-A\geq M(1-\alpha)$}
            \RETURN{Reject}
            \ENDIF
        \ENDFOR
    \end{algorithmic}
\end{algorithm}


\section*{References}

\bibliography{mybibfile}

\end{document}
